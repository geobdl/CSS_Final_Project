%% LyX 2.0.5.1 created this file.  For more info, see http://www.lyx.org/.
%% Do not edit unless you really know what you are doing.
\documentclass[english]{article}
\usepackage[T1]{fontenc}
\usepackage{listings}
\usepackage{geometry}
\geometry{verbose,tmargin=1in,bmargin=1in,lmargin=1in,rmargin=1in}
\usepackage{color}
\usepackage{babel}
\usepackage{float}
\usepackage{url}
\usepackage{graphicx}
\usepackage[authoryear]{natbib}
\usepackage[unicode=true,pdfusetitle,
 bookmarks=true,bookmarksnumbered=false,bookmarksopen=false,
 breaklinks=false,pdfborder={0 0 1},backref=false,colorlinks=true]
 {hyperref}

\makeatletter

%%%%%%%%%%%%%%%%%%%%%%%%%%%%%% LyX specific LaTeX commands.
%% Because html converters don't know tabularnewline
\providecommand{\tabularnewline}{\\}

%%%%%%%%%%%%%%%%%%%%%%%%%%%%%% User specified LaTeX commands.
\bibpunct[, ]{[}{]}{;}{a}{}{,}

\makeatother

\usepackage{xunicode}
\begin{document}

\title{APMA E4990 Final Project Report}


\author{Varun Ravishankar, vr2263 and Viktor Roytman, vr2262}

\maketitle

\section{Introduction}

It's no surprise that people value the songs of their youth and often
react with disdain at newer music. The problem with this mindset is
that it is no way to judge music; it is completely subjective. What
we would like is an objective way to quantify and compare songs and
determine who's right. As it happens, with the Million Song Dataset
we can do just that \citep{Bertin-Mahieux2011}.

We analyzed the Million Song Dataset in order to answer a few questions
we had about the state of music and how it has changed over time.
Our initial question was whether every song has the word ``baby''
in its lyrics. From this we decided to do a general analysis of word
frequency, taking into account the prevalence of common words, and
to highlight our favorite words. Next we considered the question of
how to see trends in music over time. To do this we would need some
musical qualities that are inherently quantifiable. The metrics we
decided to study are song duration, tempo, and loudness, and we investigated
the distribution and change over time for each one. By answering these
questions, we can identify trends in the state of music and determine
whether historical events have had noticeable effects on these musical
characteristics.

In addition to our ``baby'' hypothesis, we had a few predictions
for our metrics. There is a well-known phenomenon known as the ``loudness
wars,'' so we predicted the loudness of songs to increase over time.
As an aside, the measure of loudness in this case is dBFS, or decibels
relative to full scale. In other words, for a digital medium and some
encoding scheme, there is a maximum possible amplitude denoted 0 dBFS,
and amplitudes smaller than that are negative. We also predicted that
the tempo of songs would increase over time (there was no metal music
in the 1940s).


\section{Datasets}

To explore music trends over time, we used the Million Song Dataset
\citep{Bertin-Mahieux2011}. This dataset contains a collection of
audio features and metadata for one million songs from 1920 until
today, and is free for non-commercial use. This data was collected
using the Echo Nest API and a copy of the musicbrainz server in December
2010. This dataset includes information on releases, musical qualities
like tempo and loudness, and tags. This dataset is split into HDF5
files, a common format for data and scientific applications. The complete
dataset is 300 GB, but we were able to focus on a smaller 300 MB dataset,
\lstinline!msd_summary_file.h5!, that contained just the metadata
without any audio analysis, similar artists, or tags. We also used
a SQLite database (\lstinline!track_metadata.db!) containing most
of the metadata on each track, including track ID, artist, and year.
Finally, we used a list of 515,576 tracks in the dataset that also
had year information, \lstinline!track_per_year.txt!.

To explore song lyrics, we used another dataset available from the
Million Song Dataset, the musiXmatch Dataset%
\footnote{\url{http://labrosa.ee.columbia.edu/millionsong/musixmatch}%
} \citep{Bertin-Mahieux2011}. This dataset provides word counts per
song for the most common 5,000 words in the data, but only contains
the lyrics for 237,662 tracks because of copyright restrictions plus
the removal of instrumentals (defined as having three words or less)
and duplicate tracks. The word counts are stored in bag of words style,
meaning position information is lost. Since there are so many words
in the entire dataset, the authors chose to limit the dataset to just
the 5,000 most common words, and also decided to run a modified Porter-Stemmer
program on the lyrics to create the counts. The dataset comes as two
files, \lstinline!mxm_dataset_train.txt! and \lstinline!mxm_dataset_test.txt!,
for standardizing machine learning tasks. Since we were not making
predictions in our analysis, we cleaned up the training and test datasets
and then concatenated the two files together.

The data can be downloaded using the accompanying script \lstinline!download_msd.sh!.


\section{Technology and Development Environment}

Our technology stack was simple. Our programs consisted of scripts
meant to be run on POSIX machines. We tested and ran our scripts on
Fedora 18 and OS X 10.8.3, using Python 3%
\footnote{\url{http://www.python.org/}%
} and Bash%
\footnote{\url{http://www.gnu.org/software/bash/}%
}. For analyzing the lyrics dataset, we wrote scripts in Python to
read in the dataset, find the word frequencies, plot the distributions,
and output statistics on the dataset. We also used NumPy%
\footnote{\url{http://www.numpy.org/}%
} to vectorize the computations and the Python natural language processing
module NLTK%
\footnote{\url{http://nltk.org/}%
} to find stopwords.

To analyze the Million Song Dataset, we used a combination of Python,
NumPy, and pandas%
\footnote{\url{http://pandas.pydata.org/}%
}. We also used queries on a SQLite%
\footnote{\url{http://www.sqlite.org/}%
} database of just the track metadata to avoid analyzing the full 300
GB dataset. Our graphs were created using matplotlib%
\footnote{\url{http://matplotlib.org/}%
}. The database was optimized for counting and joining, and so we took
advantage of the speedups possible when querying a relational database.


\section{Lyrics}

We first calculated the frequency of common words in songs. Because
each word could appear in a single song more than once and heavily
used words could heavily skew the word distribution, we chose to count
a word occurrence only once per song. We wrote a Python script, \lstinline!read_lyrics.py!,
to read in the lyrics dataset, parse the file, store the counts in
an array, and then rank the counts. We maintained a reverse index
of word to count to enable querying word frequency.

We started with our initial hypothesis that the words ``baby'' and
``love'' occurred in most songs. However, the actual frequency of
``baby'' and ``love'' was far lower than our expectations, as
seen in Table \ref{tab:Frequency-of-baby}:

\begin{table}[H]


\begin{centering}
\begin{tabular}{|c|c|c|}
\hline 
Word & Rank & Frequency\tabularnewline
\hline 
\hline 
baby & 110 & 13.35\%\tabularnewline
\hline 
love & 37 & 30.89\%\tabularnewline
\hline 
\end{tabular}\caption{Frequency of ``baby'' and ``love''\label{tab:Frequency-of-baby}}

\par\end{centering}

\end{table}


Because these frequencies were far lower than what we expected, we
investigated the overall distribution of words and their frequencies,
as can be seen in Figure \ref{fig:Distribution-of-Lyrics}. Figure
\ref{fig:Distribution-of-Lyrics} is a histogram of the word frequencies,
and is heavily skewed. Most words have low frequencies, as indicated
by the long tail in the histogram. This is further confirmed by the
summary statistics on the lyrics distribution displayed in Table \ref{tab:Summary-Statistics-of-Lyrics}.
The mean is relatively high, but the median is only 152. This evidence
is compounded by the high standard deviation, indicating a heavily
skewed distribution.

\begin{figure}[H]
\centering{}\includegraphics[scale=0.6]{data/graphs/lyrics_histogram}\caption{Distribution of Lyrics \label{fig:Distribution-of-Lyrics}}
\end{figure}


\begin{table}[H]
\centering{}%
\begin{tabular}{|c|c|}
\hline 
Statistic & Measure\tabularnewline
\hline 
\hline 
Count & 5000\tabularnewline
\hline 
Mean & 3809.07\tabularnewline
\hline 
Standard Deviation & 12624.51\tabularnewline
\hline 
Min & 73\tabularnewline
\hline 
25\% & 120.0\tabularnewline
\hline 
50\% & 151.99\tabularnewline
\hline 
75\% & 179.49\tabularnewline
\hline 
Max & 188401\tabularnewline
\hline 
\end{tabular}\caption{Summary Statistics of Lyrics Distribution\label{tab:Summary-Statistics-of-Lyrics}}
\end{table}


Because of the skew in the data, we decided to look at the top 10
words to get a sense of the highest frequencies. This data, shown
in Table \ref{tab:Frequency-of-Top}, indicates that the top 10 words
in the lyrics distribution are mostly stopwords, a term used in natural
language processing to indicate words like articles and common prepositions
that typically end a sentence and carry no special meaning. These
stopwords occur in about 60\%-70\% of the data. Because they are so
common, we decided that it was worth looking at terms that were not
stopwords.

\begin{table}[H]
\centering{}%
\begin{tabular}{|c|c|c|}
\hline 
Rank & Word & Frequency\tabularnewline
\hline 
\hline 
1 & the & 79.27\%\tabularnewline
\hline 
2 & a & 77.19\%\tabularnewline
\hline 
3 & to & 75.43\%\tabularnewline
\hline 
4 & and & 74.02\%\tabularnewline
\hline 
5 & i & 73.25\%\tabularnewline
\hline 
6 & you & 70.17\%\tabularnewline
\hline 
7 & in & 67.36\%\tabularnewline
\hline 
8 & is & 64.63\%\tabularnewline
\hline 
9 & me & 63.58\%\tabularnewline
\hline 
10 & it & 62.91\%\tabularnewline
\hline 
\end{tabular}\caption{Frequency of Top 10 Words\label{tab:Frequency-of-Top}}
\end{table}


Using a list of stopwords provided by NLTK, we found the 10 most common
words that were not stopwords, as seen in Table \ref{tab:Frequency-of-Top-Not-Stop}.
This indicates that much of the top 50 words are stopwords, and that
the rest of the common words only occur in about 30\%-40\% of the
data.

\begin{table}[H]
\centering{}%
\begin{tabular}{|c|c|c|}
\hline 
Rank & Word & Frequency\tabularnewline
\hline 
\hline 
29 & know & 38.56\%\tabularnewline
\hline 
35 & like & 33.34\%\tabularnewline
\hline 
37 & time & 31.50\%\tabularnewline
\hline 
38 & love & 30.89\%\tabularnewline
\hline 
43 & see & 29.39\%\tabularnewline
\hline 
44 & come & 28.83\%\tabularnewline
\hline 
45 & go & 28.82\%\tabularnewline
\hline 
46 & one & 28.48\%\tabularnewline
\hline 
50 & get & 26.74\%\tabularnewline
\hline 
53 & never & 25.18\%\tabularnewline
\hline 
\end{tabular}\caption{Frequency of Top 10 Words that are not Stopwords\label{tab:Frequency-of-Top-Not-Stop}}
\end{table}


Finally, we were curious about the prevalence of swear words in music
after anecdotally observing an increase in both the acceptance and
occurrence of swear words in today's music. We created a list of colorful
language and found their frequencies, as shown in Table \ref{tab:Frequency-of-Top-Swear}.
Surprisingly, curse words occur in only 1\%-5\% of all songs. However,
with the knowledge that this dataset contains 5000 words, this is
still surprisingly high amount. The first curse word is only Rank
294, while the last one in the table is Rank 1286.

\begin{table}[H]
\centering{}%
\begin{tabular}{|c|c|c|}
\hline 
Rank & Word & Frequency\tabularnewline
\hline 
\hline 
294 & fu{*}{*} & 5.17\%\tabularnewline
\hline 
333 & hell & 4.47\%\tabularnewline
\hline 
360 & shit & 4.14\%\tabularnewline
\hline 
416 & blow & 3.51\%\tabularnewline
\hline 
593 & ass & 2.26\%\tabularnewline
\hline 
614 & b{*}tch & 2.14\%\tabularnewline
\hline 
628 & n{*}{*}ga & 2.08\%\tabularnewline
\hline 
936 & fu{*}{*}in & 1.30\%\tabularnewline
\hline 
1093 & sex & 1.06\%\tabularnewline
\hline 
1286 & d{*}{*}k & 0.88\%\tabularnewline
\hline 
\end{tabular}\caption{Frequency of Top 10 Swear Words\label{tab:Frequency-of-Top-Swear}}
\end{table}



\section{Duration}

The \lstinline!track_metadata.db! file includes a duration for all
of the million songs and a year for about half of them. We collected
the data by nesting a query for the count and average duration of
songs by year in a query for the year and average duration where the
count is at least 10 and the year information is present. We then
created a line graph of average duration as a function of year, seen
in Figure \ref{fig:Average-Song-Duration}. There is a very clear
jump in average duration around 1970.

\begin{figure}[H]
\begin{centering}
\includegraphics[scale=0.6]{data/graphs/graph_duration_v_time}\caption{Average Song Duration Over Time\label{fig:Average-Song-Duration}}

\par\end{centering}

\end{figure}


The creation of the duration histogram is even more straightforward:
it is just the binning of a query for duration. This histogram can
be seen in Figure \ref{fig:Song-Duration-Distribution}. There are
very few songs with exceptionally long duration, and we had no success
confirming that the songs with the longest duration in the dataset
were actually that long (30 - 60 minutes). The summary statistics
for this distribution, displayed in Table \ref{tab:Summary-Statistics-of-Duration},
verify this and tell us that there is a high spread but little variation
for most songs.

\begin{figure}[H]
\centering{}\includegraphics[scale=0.6]{data/graphs/graph_duration_histogram}\caption{Song Duration Distribution\label{fig:Song-Duration-Distribution}}
\end{figure}


\begin{table}[H]
\centering{}%
\begin{tabular}{|c|c|}
\hline 
Statistic & Measure\tabularnewline
\hline 
\hline 
Count & 1000000\tabularnewline
\hline 
Mean & 4.16 min\tabularnewline
\hline 
Standard Deviation & 2.10 min\tabularnewline
\hline 
Min & 0.0052 min\tabularnewline
\hline 
25\% & 0.30 min\tabularnewline
\hline 
50\% & 0.45 min\tabularnewline
\hline 
75\% & 0.58 min\tabularnewline
\hline 
Max & 50.58 min\tabularnewline
\hline 
\end{tabular}\caption{Summary Statistics of Duration Distribution\label{tab:Summary-Statistics-of-Duration}}
\end{table}



\section{Loudness}

The loudness information is in a summary HDF5 file, \lstinline!msd_summary_file.h5!.
We pick out only the relevant columns, which are \lstinline!track_id!
and \lstinline!loudness!. We then step through another file, \lstinline!tracks_per_year.txt!,
assembling lists of loudnesses per year. Finally, we filter out years
with fewer than 10 songs, and take the average of the loudnesses.
The resulting graph clearly shows an upward trend in loudness over
time, and can be seen in Figure \ref{fig:Average-Loudness-per}.

\begin{figure}[H]
\centering{}\includegraphics[scale=0.6]{data/graphs/graph_loudness_v_time}\caption{Average Loudness per Year \label{fig:Average-Loudness-per}}
\end{figure}


The loudness histogram, seen in Figure \ref{fig:Song-Loudness-Distribution},
is a simple binning of the loudness information from the summary file.
It is skewed heavily toward a loudness of 0, since most of the songs
in the dataset are recent, and the more recent songs tend to be louder.
This is further confirmed by Table \ref{tab:Summary-Statistics-of-Loudness},
which displays the summary statistics for loudness. We can see that
the distribution is heavily skewed by the fact that the mean and median
are so far apart, with further evidence given by the large standard
deviation and large extremes given by the min and max.

\begin{figure}[H]
\centering{}\includegraphics[scale=0.6]{data/graphs/graph_loudness_histogram}\caption{Song Loudness Distribution \label{fig:Song-Loudness-Distribution}}
\end{figure}


\begin{table}[H]
\centering{}%
\begin{tabular}{|c|c|}
\hline 
Statistic & Measure\tabularnewline
\hline 
\hline 
Count & 1000000\tabularnewline
\hline 
Mean & -10.12 dBFS\tabularnewline
\hline 
Standard Deviation & 5.20 dBFS\tabularnewline
\hline 
Min & -58.18 dBFS\tabularnewline
\hline 
25\% & -33.03 dBFS\tabularnewline
\hline 
50\% & -30.26 dBFS\tabularnewline
\hline 
75\% & -28.54 dBFS\tabularnewline
\hline 
Max & 4.32 dBFS\tabularnewline
\hline 
\end{tabular}\caption{Summary Statistics of Loudness Distribution\label{tab:Summary-Statistics-of-Loudness}}
\end{table}



\section{Tempo}

The acquisition method of the tempo information is very similar to
that of the loudness information, since it is also in the summary
HDF5 file. The resulting graph, seen in Figure \ref{fig:Average-Tempo-per},
shows wide swings in average tempo for the early years in the dataset,
with a severe leveling off in recent years. It is not clear why there
are such swings for the early years. It could be an artifact of the
dataset or a sign of fashions in music changing quickly. In any case,
generally tempo has increased with time.

\begin{figure}[H]
\centering{}\includegraphics[scale=0.6]{data/graphs/graph_tempo_v_time}\caption{Average Tempo per Year \label{fig:Average-Tempo-per}}
\end{figure}


The histogram, seen in Figure \ref{fig:Song-Tempo-Distribution},
shows something like a normal distribution for tempo, with a longer
tail of high tempo songs. We excluded songs with a tempo of 0, since
we considered those particular songs to be untagged. However, like
duration, we had no success confirming that the songs with the highest
or lowest tempos were catalogued correctly.

The summary statistics for the distribution, shown in Table \ref{tab:Summary-Statistics-of-Tempo},
show a high spread for the values and show that while the distribution
looks roughly Gaussian, is very skewed. A high standard deviation
only confirms this.

\begin{figure}[H]
\centering{}\includegraphics[scale=0.6]{data/graphs/graph_tempo_histogram}\caption{Song Tempo Distribution \label{fig:Song-Tempo-Distribution}}
\end{figure}


\begin{table}[H]
\centering{}%
\begin{tabular}{|c|c|}
\hline 
Statistic & Measure\tabularnewline
\hline 
\hline 
Count & 996770\tabularnewline
\hline 
Mean & 124.29 bpm\tabularnewline
\hline 
Standard Deviation & 34.39 bpm\tabularnewline
\hline 
Min & 7.36 bpm\tabularnewline
\hline 
25\% & 39.77 bpm\tabularnewline
\hline 
50\% & 45.02 bpm\tabularnewline
\hline 
75\% & 50.00 bpm\tabularnewline
\hline 
Max & 302.3\tabularnewline
\hline 
\end{tabular}\caption{Summary Statistics of Tempo Distribution\label{tab:Summary-Statistics-of-Tempo}}
\end{table}



\section{Future Work}

In the future, we would like to focus on genre analysis. We originally
planned to use Amazon EC2, Amazon EMR, and Hadoop to explore the Million
Song Dataset. We planned to find genres and then calculate duration,
tempo, volume, and other features per genre. This did not work out
because when we explored some of the HDF5 files in the dataset, the
genres were set to empty strings. This was true even when we explored
the full dataset. To further complicate matters, the dataset did not
list genre as a valid field. Instead, it seems that we would have
had to explore the musicbrainz tags, which are stored as arrays of
strings per song. Because these tags are only a weak indicator of
genre and because we tried to avoid looking at the full dataset as
much as possible, we decided it would be best to explore an alternative
dataset. We began to look at the Last.fm dataset%
\footnote{\url{http://labrosa.ee.columbia.edu/millionsong/lastfm}%
} \citep{Bertin-Mahieux2011}, but ran out of time before we could
explore this fully. We also wanted to look at lyrics distribution
over time, but the format of the datasets did not make that straightforward.
Given more time, we would look into the frequencies of words used
over time, whether the top words changed over time, and whether words
such as curse words starting becoming more prevalent in the 80s and
90s.


\section{Conclusion}

Most of our hypotheses were somewhat off the mark. While ``love,''
discounting the stopwords, is one of the most common words, ``baby''
is much further down in rank, contrary to our expectations. The most
common of these words are ``know,'' ``like,'' and ``time.''
In addition, fu{*}{*} is a surprisingly common word in songs, occurring
in over 5\% of all tracks.

Our analysis of the metrics reveal some interesting discoveries. There
is an obvious jump in duration around 1970, indicative of better music
storage technology, and most songs don't deviate far from around 4
minutes in duration. Loudness increases more or less steadily over
the years, and recently it has been getting rather close to 0 dBFS.
Surprisingly, there are large swings in average tempo up until the
1960s or so, but after around 1990 the average tempo levels off. This
may be indicative of a trend toward greater homogeneity in songs in
recent years.

\bibliographystyle{ACM-Reference-Format-Journals}
\bibliography{Final_Project_Report}

\end{document}
